\documentclass{article}

\usepackage{fancyhdr}
\usepackage{extramarks}
\usepackage{amsmath}
\usepackage{amsthm}
\usepackage{amsfonts}
\usepackage{tikz}
\usepackage[plain]{algorithm}
\usepackage{algpseudocode}
\usepackage{xepersian}
 \usepackage{xcolor}
 \usepackage{lscape}
 \renewcommand{\refname}{\rl{{مراجع}\hfill}}
\settextfont[Scale=1.2]{XB Niloofar}
%\setlatintextfont{Serif}
\setdigitfont[Scale=1.2]{Yas}
\DefaultMathsDigits


\usetikzlibrary{automata,positioning}

%
% Basic Document Settings
%

\topmargin=-0.35in
\evensidemargin=0in
\oddsidemargin=0in
\textwidth=6.5in
\textheight=9.0in
\headsep=0.25in

\linespread{1.1}

\pagestyle{fancy}
\lhead{بهراد منیری
(۹۵۱۰۹۵۶۴)
}
\rhead{تمرین سری هفتم درس بهینه‌سازی عددی}


\renewcommand\headrulewidth{0.4pt}

\setlength\parindent{0pt}

\title{جداسازی کور ترکیب‌های غیرخطی فرآیند‌های تصادفی}
\author{
	بهراد منیری\\
	\texttt{bemoniri@ee.sharif.edu}
	\vspace{0.5cm}\\
	استاد راهنما:
	دکتر مسعود بابائی‌زاده
}
\date{}

\renewcommand{\part}[1]{\textbf{\large Part \Alph{partCounter}}\stepcounter{partCounter}\\}

%
% Various Helper Commands
%

% Useful for algorithms
\newcommand{\alg}[1]{\textsc{\bfseries \footnotesize #1}}


\newtheorem{den}{{\large\bf تعریف}}[section]
\newtheorem{exa}{{\large\bf مثال}}[section]
\newtheorem{lem}{{\large\bf لم}}[section]
\newtheorem{pro}{{\large\bf گزاره}}[section]
\newtheorem{cor}{{\large\bf نتیجه}}[section]
\newtheorem{thm}{{\large\bf قضیه}}[section]
\newtheorem{rem}{{\large\bf تذکر}}[section]
\newtheorem{nnt}{{\large\bf توجه}}[section]
\newtheorem{aaa}{{\large\bf}}[section]

\begin{document}

\begin{landscape}

 \section{تکرار تمرین چهارم با شرایط ولف}

در این تمرین، الگوریتم
\lr{linesearch}
بر مبنای شرایط قوی ولف پیاده‌سازی شده و تمرین شماره‌ی ۴ مجدداً با استفاده از این 
\lr{linesearch}
انجام می‌شود. دو روش 
\lr{linesearch}
استفاده از 
\lr{GSS}
و روش مبتنی بر شرایط قوی ولف، در جداول زیر مقایسه شده‌اند. این مقایسه برای دو تابع پاول و روزنبرک انجام شده است. مطابق تمرین شماره‌ی چهار، همواره
\lr{stop\_tol}
برابر
$10^{-4}$
در نظر گرفته‌ شده‌است. شرایط اولیه نیز مطابق تمرین چهارم است.
 کمیت‌های مورد استفاده برای 
\lr{linesearch}
نیز به شرح زیر است:
\begin{equation*}
\begin{cases}
c_1 = 10^{-4}\\
c_2 = 0.1
\end{cases}
\end{equation*}


در مقایسه‌های زیر، مشاهده می‌شود که عملکرد الگوریتم
\lr{Wolfe}
بسیار بهتر از الگوریتم
\lr{GSS}
بوده است. این امر، به وضوح در عملکرد الگوریتم 
\lr{Steepest Descent}
بر روی تابع روزنبروک قابل مشاهده است. در جالتی که از الگوریتم 
\lr{GSS}
استفاده می‌شد، جواب بسیار بسیار تابع پارامتر‌های
\lr{linesearch}
بود. این موضوع در استفاده از شرایط ولف، بسیار بهتر شده است و الگوریتم نسبت به این پارامتر‌ها، بسیار پایدار است. در سه آزمایش دیگر نیز، شرایط ولف، جواب بهتری به نسبت روش
\lr{GSS}
داده‌اند. متطابق الگوریتم مطرح شده در کلاس، در هر مرحله‌ از لاین‌سرچ، ابتدا 
$\alpha = 1$
چک شده‌است.
\vspace{3cm}	

\begin{latin}
\begin{table}[h!]
	\begin{tabular}{|c|c|c|c|c|c|c|c|}
		\hline
		
		& Linesearch & \textbf{Final x}  & Final f    & \# Iter & \# Func Evals  &\# Grad Evals & \# Hess. Evals \\ \hline
											
		Rosenbrock Function  (SD)     & GSS    & {[}1.2733, 1.6215{]}                                          & 0.0747     & 105     & 3781          & 105           & 0              \\	
			
		Rosenbrock Function  (SD)     & Wolfe    & {[}1.3506 1.8255{]}                                          &      0.12311
		& 3     & 127         & 54          & 0              \\	
		
		 \hline
		 
		Rosenbrock Function (ٔNewton) & GSS    & {[}1.0000, 1.0000{]}                                          & 2.2542e-19 & 2       & 48            & 34             & 2              \\ 
		
		Rosenbrock Function (ٔNewton)  & Wolfe   & {[}1.0000, 1.0000{]}                                          & 0.0000e-21 & 2       & 7            & 4             & 2              \\ 
		
		\hline
		
		Powell Function (SD)  & GSS    & {[}0.2302, -0.230, 0.1101, 0.1190{]}                          & 0.0054     & 181     & 6517          & 181           & 0              \\
		
		Powell Function (SD) & Wolfe      & {[}   0.2388, -0.0237, 0.1139,   0.1237{]}                          & 0.00624    & 213     & 13693          & 8051           & 0              \\	
		 \hline
		
		Powell Function (Newton) & GSS & 10\textasciicircum{}-3{[}-0.0001, -0.0002, -0.8448, 0.8448{]} & 1.7404e-11 & 11      & 397           & 11            & 11             \\ 
		
		Powell Function (Newton) & Wolfe & {[}   0.0000,  -0.0000,  0.0011, 0.0011 {]} & 0.001083e-11 & 12      & 45           & 35            & 12             \\ 
		
		
		\hline
	\end{tabular}
\end{table}
\end{latin}

\end{landscape}
\end{document}
