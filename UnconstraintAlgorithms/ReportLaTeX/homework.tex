\documentclass{article}

\usepackage{fancyhdr}
\usepackage{extramarks}
\usepackage{amsmath}
\usepackage{amsthm}
\usepackage{amsfonts}
\usepackage{tikz}
\usepackage[plain]{algorithm}
\usepackage{algpseudocode}
\usepackage{xepersian}
 \usepackage{xcolor}
 \usepackage{lscape}
 \renewcommand{\refname}{\rl{{مراجع}\hfill}}
\settextfont[Scale=1.2]{XB Niloofar}
%\setlatintextfont{Serif}
\setdigitfont[Scale=1.2]{Yas}
\DefaultMathsDigits


\usetikzlibrary{automata,positioning}

%
% Basic Document Settings
%

\topmargin=-0.35in
\evensidemargin=0in
\oddsidemargin=0in
\textwidth=6.5in
\textheight=9.0in
\headsep=0.25in

\linespread{1.1}

\pagestyle{fancy}
\lhead{بهراد منیری
(۹۵۱۰۹۵۶۴)
}
\rhead{تمرین سری چهارم درس بهینه‌سازی عددی}
\lfoot{\lastxmark}
\cfoot{\thepage}

\renewcommand\headrulewidth{0.4pt}
\renewcommand\footrulewidth{0.4pt}

\setlength\parindent{0pt}

\title{جداسازی کور ترکیب‌های غیرخطی فرآیند‌های تصادفی}
\author{
	بهراد منیری\\
	\texttt{bemoniri@ee.sharif.edu}
	\vspace{0.5cm}\\
	استاد راهنما:
	دکتر مسعود بابائی‌زاده
}
\date{}

\renewcommand{\part}[1]{\textbf{\large Part \Alph{partCounter}}\stepcounter{partCounter}\\}

%
% Various Helper Commands
%

% Useful for algorithms
\newcommand{\alg}[1]{\textsc{\bfseries \footnotesize #1}}


\newtheorem{den}{{\large\bf تعریف}}[section]
\newtheorem{exa}{{\large\bf مثال}}[section]
\newtheorem{lem}{{\large\bf لم}}[section]
\newtheorem{pro}{{\large\bf گزاره}}[section]
\newtheorem{cor}{{\large\bf نتیجه}}[section]
\newtheorem{thm}{{\large\bf قضیه}}[section]
\newtheorem{rem}{{\large\bf تذکر}}[section]
\newtheorem{nnt}{{\large\bf توجه}}[section]
\newtheorem{aaa}{{\large\bf}}[section]

\begin{document}
%\maketitle
\begin{landscape}
\section{توضیح مختصر}
در این تمرین، پاسخ‌ها را با آقای امیرحسین افشارراد چک کردم. جواب‌های تا حد بسیار زیادی شبیه‌ به هم بود، مگر در تابع روزنبروک و در حالت استفاده از الگوریتم 
\lr{SD}.
در این تمرین، در تمام الگوریتم‌ها، لاین‌سرچ انجام می‌شود که 
\lr{GSS}
مربوط به این لاین‌سرچ‌ها، همگی در بازه‌ی 
$[0, 100]$
انجام می‌شود. تابع ارسالی توسط من، برای 
\lr{GSS}
در تمرین سری اول به صورت بازگشتی پیاده‌سازی شده بود. در این‌جا با پیاده‌سازی مجدد و غیربازگشتی آن تابع، جواب آن بخش نیز شبیه به جواب آقای افشارراد شد. حالت اول، حالتی است که از همان  
\lr{GSS}
تمرین اول استفاده کرده‌ام و حالت دوم، حالتی است که از 
\lr{GSS}
جدید استفاده کرده‌ام. دلیل این تفاوت هم اختلاف بسیار بسیار اندک در 
\lr{GSS}
است (از اردر 
$10^{-14}$
برای $\alpha$ در هر ایتریشن) که به دلیل حساسیت بسیار زیاد تابع روزنبروک، این مسئله بسیار شدید نشان داده‌شده است. در فایل زیپ‌ ارسالی، دو ابع اصلی این تمرین به صورت مجزا با نام های
\lr{Newton\_GSS.m}
و 
\lr{SD\_GSS.m}
آورده شده‌است.
(توجه کنید که در فایل زیپ ارسالی دو فولدر 
\lr{Method1}
و
\lr{Method2}
نیز وجود دارند که کد‌های کامل هر حالت در آن یافت می‌شود و جواب‌های زیر را تولید می‌کند.)

\section{حالت اول}
% Please add the following required packages to your document preamble:
% \usepackage[table,xcdraw]{xcolor}
% If you use beamer only pass "xcolor=table" option, i.e. \documentclass[xcolor=table]{beamer}
\begin{latin}
\begin{table}[h!]
	\begin{tabular}{|c|c|c|c|c|c|c|}
		\hline
		& \textbf{Final x}                                              & Final f    & \# Iter & \# Func Evals & \# Grad Evals & \# Hess. Evals \\ \hline
		Powel Function  (SD)         & {[}1.2733, 1.6215{]}                                          & 0.0747     & 105     & 3781          & 105           & 0              \\ \hline
		Rosenbrock Function (SD)     & {[}1.0000, 1.0000{]}                                          & 2.2542e-19 & 2       & 63            & 2             & 2              \\ \hline
		Powel Function (Newton)      & {[}0.2302, -0.230, 0.1101, 0.1190{]}                          & 0.0054     & 181     & 6517          & 181           & 0              \\ \hline
		Rosenbrock Function (Newton) & 10\textasciicircum{}-3{[}-0.0001, -0.0002, -0.8448, 0.8448{]} & 1.7404e-11 & 11      & 397           & 11            & 11             \\ \hline
	\end{tabular}
\end{table}
\end{latin}

\section{حالت دوم}
% Please add the following required packages to your document preamble:
% \usepackage[table,xcdraw]{xcolor}
% If you use beamer only pass "xcolor=table" option, i.e. \documentclass[xcolor=table]{beamer}
\begin{latin}
\begin{table}[h!]
	\begin{tabular}{|c|c|c|c|c|c|c|}
		\hline
		& \textbf{Final x}                                              & Final f    & \# Iter & \# Func Evals & \# Grad Evals & \# Hess. Evals \\ \hline
		Powel Function  (SD)         & {[}1.0033, 1.0067{]}                                          & 1.1050e-05 & 184     & 12513         & 184           & 0              \\ \hline
		Rosenbrock Function (SD)     & {[}1.0000, 1.0000{]}                                          & 2.2542e-19 & 2       & 117           & 2             & 2              \\ \hline
		Powel Function (Newton)      & {[}0.2302, -0.230, 0.1101, 0.1190{]}                          & 0.0054     & 181     & 12309         & 181           & 0              \\ \hline
		Rosenbrock Function (Newton) & 10\textasciicircum{}-3{[}-0.0001, -0.0002, -0.8448, 0.8448{]} & 1.7404e-11 & 11      & 794           & 11            & 11             \\ \hline
	\end{tabular}
\end{table}
\end{latin}
\end{landscape}
\end{document}
